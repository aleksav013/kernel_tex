\documentclass[a4paper,fleqn,12pt]{JMThesis}
\usepackage{listings}
\lstset{
    basicstyle=\footnotesize,
    frame=single,
    showstringspaces=false,
    tabsize=4,
    escapeinside={<@}{@>},
%    numbers=left
}
\lstset{defaultdialect=[x86masm]Assembler}

\usepackage[OT2]{fontenc}
\newcommand\eng{\fontencoding{OT1}\fontfamily{\rmdefault}\selectfont}
\newcommand\srb{\fontencoding{OT2}\fontfamily{\rmdefault}\selectfont}

% \usepackage[serbian,english]{babel}

\renewcommand{\baselinestretch}{1}
\usepackage{amsfonts}
\usepackage{amsthm}
\usepackage{amsmath}
\usepackage{multicol}
\usepackage{hyperref}
\usepackage{tocloft}
\usepackage[
    backend=bibtex,
    natbib=true,
    style=numeric,
    block=ragged,
    sorting=none
]{biblatex}
\bibliography{kernel}
\renewcommand*{\bibfont}{\eng}

\newlength\tindent
\setlength{\tindent}{\parindent}
\setlength{\parindent}{0pt}
\setlength{\itemsep}{0pt}
\setlength{\parskip}{0pt}
\setlength{\parsep}{0pt}
\renewcommand{\indent}{\hspace*{\tindent}}

\oddsidemargin 1cm
\evensidemargin 0cm
\textwidth 15cm

\pagestyle{headings}

\font \matematicka=wncsc10 scaled 1600
\font \maturski=wncyb10 scaled 2500
\font \naslov=wncyb10 scaled 1900
\font \naslovlat=cmr10 scaled 1900
\font \imen=wncyr10 scaled 1600


\def\zn{,\kern-0.09em,}
\def\zng{'\kern-0.09em'}
\pagenumbering{roman}
\begin{document}


\thispagestyle{empty}

\begin{center}
{\matematicka Matematichka gimnazija}
\end{center}
\vspace*{50mm}

\begin{center}
{\maturski MATURSKI RAD}

\vspace*{8pt}
{\naslov - iz rachunarstva i informatike -}
\end{center}

\vspace*{10pt}
\begin{center}
    {\naslov Izrada \textbf{\eng\Large X86 32bit i686} jezgra operativnog sistema}
\end{center}

\vspace*{70mm}
\setlength{\columnsep}{50pt}
\begin{multicols}{2}
 {\noindent \imen Uchenik:
\\Aleksa Vuchkovic1  $\operatorname{IV}$d}


{ \noindent \hfill \imen Mentor:\\
\hfill \phantom{aaaaaaaa} Milosh Arsic1}
\end{multicols}

\vfill
\begin{center}
{\imen Beograd, jun 2021.}
\end{center}
\clearpage

\thispagestyle{empty}
\mbox{}
\clearpage


\renewcommand{\contentsname}{Sadrzhaj}
\thispagestyle{empty}

\pagenumbering{gobble}

\tableofcontents \clearpage

\thispagestyle{empty}
\mbox{}
\clearpage

\pagenumbering{arabic}
\renewcommand{\chaptername}{}
\setcounter{page}{1}

%%%%%%%%%%%%%%%%%%%%%%%%%%%%%%%%%%%%%%%%%%%%%%%%%%%%%%%%%%%%%%%%%%%%%%%%%%%%%%%%
%%%%%%%%%%%%%%%%%%%%%%%%%%%%%%%%%%%%%%%%%%%%%%%%%%%%%%%%%%%%%%%%%%%%%%%%%%%%%%%%
%%%%%%%%%%%%%%%%%%%%%%%%%%%%%%%%%%%%%%%%%%%%%%%%%%%%%%%%%%%%%%%%%%%%%%%%%%%%%%%%
\chapter{Uvod}
\bigskip

Ideja za ovaj rad prozishla je iz ekstenzivnog korish\/c1enja {\eng GNU/Linux}
sistema, kao i zhelja za razumevanjem rada rachunara na najnizhem nivou.

Ceo kod je pisan u {\eng GNU Asembler}-u i {\eng C}-u i mozhe se nac1i na
{\eng GitHub}-u na stranici {\eng\url{https://github.com/aleksav013/mykernel}}.
Kod, zajedno sa svim alatima za njegovo korishenje i upotrebu, je dostupan pod
{\eng GPLv3} licencom.

Ovaj projakat se konstantno unapredjuje i nemoguc1e je odrzhavati
sinhronizovanim fajlove koji sachinjavaju operativni sistem, a koji se
istovremeno pominju u ovom radu. Iz tog razloga, rad c1e biti napisan za verziju
operativnog sistema 1.0.0. Na stranici {\eng GitHub}-a ova verzija se mozhe
nac1i pod {\eng tag/s} sekcijom na stranici projekta.

Radi laksheg kompajlovanja koda od strane chitaoca, kreiran je {\eng toolchain}
- set alata, specifichno za kompajlovanje ovog operativnog sistema, koji se mozhe
nac1i na stranici {\eng\url{https://github.com/aleksav013/aleksa-toolchain}},
takodje dostupan pod {\eng GPLv3} licencom otvorenog koda. Set alata {\eng
aleksa-toolchain} kreiran je takodje i iz razloga shto je zapravo neizbezhan
proces kreiranja {\eng cross-compilera} - kompajlera koji kompapajluje kod za
drugi sistem/arhitekturu na trenutnom sistemu/arhitekturi. Na taj nachin i
drugi ljudi osim autora mogu doprineti izradi i pobojshanju ovog operativnog
sistema u buduc1nosti.\\

Ovaj projakat ima za cilj da prikazhe postupak kreiranja jednog za sada
vrlo jednostavnog operativnog sistema, kao i da potkrepi chitaoce teorijom
potrebnom za njegovo razumevanje.

%%%%%%%%%%%%%%%%%%%%%%%%%%%%%%%%%%%%%%%%%%%%%%%%%%%%%%%%%%%%%%%%%%%%%%%%%%%%%%%%
%%%%%%%%%%%%%%%%%%%%%%%%%%%%%%%%%%%%%%%%%%%%%%%%%%%%%%%%%%%%%%%%%%%%%%%%%%%%%%%%
%%%%%%%%%%%%%%%%%%%%%%%%%%%%%%%%%%%%%%%%%%%%%%%%%%%%%%%%%%%%%%%%%%%%%%%%%%%%%%%%
\chapter{{\eng X86} arhitektura}
\bigskip

{\eng X86} arhitektura je probitno bila osmobitna (sadrzhala je registre duzhine 8
bitova), 16bitna, zatim 32bitna i na kraju 64bitna. Danas 64bitnu {\eng X86}
arhitekturu znamo kao i {\eng{} AMD64}, {\eng X86-64} ili {\eng X86\_64}.\\

Zajedno sa {\eng ARM}-om jedna od najkorish\/c1enijih arhitektura danashnjice.

\section{Registri procesora}
\medskip

Postoji vishe vrsta registara procesora\cite[75]{misc:1}. Neki od osnovnih
registara koje je potrebno pomenuti dati su u tekstu koji sledi. Razlog zbog
kojeg su navedena i imena registara prethodnih verzija {\eng X86} arhitekture
je zbog toga shto je moguc1e adresirati prvih {\eng x} bitova ako se koristi ime
registra za {\eng x}-tobitnu verziju {\eng X86} arhitekture. Naravno, ovo vazhi samo
ukoliko je duzhina registra vec1a ili jednaka duzhini registra chiju notaciju
koristimo.\\

Registri opshte namene:\\[1mm]
{\eng\begin{tabular}{|c|c|c|c|c|c|c|c|c|}
\hline
8bit & al & bl & cl & dl & sil & dil & spl & bpl \\
\hline
16bit & ax & bx & cx & dx & si & di & sp & bp \\
\hline
32bit & eax & ebx & ecx & edx & esi & edi & esp & ebp \\
\hline
64bit & rax & rbx & rcx & rdx & rsi & rdi & rsp & rdp \\
\hline
\end{tabular}}\\

Segmentni registri:\\[1mm]
{\eng\begin{tabular}{|c|c|c|c|c|c|}
\hline
cs & ds & ss & es & fs & gs \\
\hline
\end{tabular}}\\

Kontrolni registri:\\[1mm]
{\eng\begin{tabular}{|c|c|c|c|c|}
\hline
cr0 & cr2 & cr3 & cr4 & cr8 \\
\hline
\end{tabular}}\\

Sistemski registri (pokazivachi na tabele):\\[1mm]
{\eng\begin{tabular}{|c|c|c|}
\hline
gdtr & ldtr & idtr \\
\hline
\end{tabular}}\\

Osim pomenutih, pochev od 16bit-ne {\eng X86} arhitekture postoje i registri
{\eng ah, bh, ch, dh (h-higher)} koji predstavljaju gornju polovinu (od 9. do
16. bita) {\eng ax, bx, cx, dx} registara redom. U daljem tekstu bic1e prikazan
odnos izmedju registara o kome je ranije bilo rechi, kao i prikaz i na to od
kog do kog bita registra se odnosi data notacija.\\

{\eng\begin{tabular}{|c|c|c|c|c|c|c|c|}
63-56 & 55-48 & 47-40 & 39-32 & 31-24 & 23-16 & 15-8 & 7-0 \\
\hline
&&&&&& ah & al \\
\hline
&&&&&& \multicolumn{2}{|c|}{ax} \\
\hline
&&&& \multicolumn{4}{|c|}{eax} \\
\hline
\multicolumn{8}{|c|}{rax} \\
\hline
\end{tabular}}\\

Primetimo da ukoliko nas interesuje vrednost drugog bajta u 64bitnoj {\eng X86}
arhitekturi, do nje mozhemo doc1i na 4 nachina:
{\eng ah, ax\&0xFF00, eax\&0x0000FF00} ili {\eng rax\&0x000000000000FF00}.


\section{Registri opshte namene}
\medskip

Registri opshte namene imaju ulogu u chuvanju operandi i pokazivacha:
\begin{itemize}
\item Operandi za logichke i aritmeticke operacije
\item Operandi za adresne kalkulacije
\item Pokazivache na memorijsku lokaciju
\end{itemize}

Registri opshte namene se mogu koristiti proizvoljno prema potrebi. Medjutim,
dizajneri hardvera su uvideli da postoji moguc1nost daljih optimizacija ukoliko
se svakom od ovih registara dodeli neka specifichna uloga u kojoj je malo bolji
od ostalih registara opshte namene.\\

Na taj nachin kompajleri vec1inu vremena kreiraju bolji asemblerski kod nego
ljudi, prosto iz razloga jer svaki od registara opshte namene takodje koriste i
za njihovu specificnu funkciju svaki put gde je to moguc1e.

Specifichna uloga registara opshte namene:
\begin{itemize}
\item {\eng eax} - akumulator za operande i podatke rezultata
\item {\eng ebx} - pokazivach na podatke u {\eng ds} segmentu
\item {\eng ecx} - brojach za petlje i operacije nad stringovima
\item {\eng edx} - pokazivach na U/I
\item {\eng esi} - pokazivach na podatke na koji pokazuje {\eng ds} registar;
    pochetni pokazivach za operacije nad stringovima
\item {\eng edi} - pokazivach na podatke u segmentu na koji pokazuje {\eng es}
    registar; krajnji pokazivach za operacije nad stringovima
\item {\eng esp} - pokazivach na pochetak steka
\item {\eng ebp} - pokazivach na podatke u steku

\end{itemize}

\section{Segmentni registri}
\medskip

Segmentni registri sadrzhe 16bitne selektore segmenta. Selektor segmenta je
specijalan pokazivach koji identifikuje segment u memoriji. Da bi pristupili
odredjenom segmentu u memoriji, selektor segmenta koji pokazuje na taj segment
mora biti dostupan u odgovarajuc1em segmentnom registru.

Specifichna uloga segmentnih registara:
\begin{itemize}
\item {\eng cs - code segment}. {\eng cs} registar sadrzhi selektor segmenta
    koji pokazuje na segment koda u kome se nalaze instrukcije koje se
    izvrshavaju.
\item {\eng ds - data segment}. Osim {\eng ds}, segmentni registri za segmente
    podataka su i {\eng es, fs}, kao i {\eng gs}.
\item {\eng ss - stack segment} {\eng ss} registar sadrzhi selektor segmenta
    koji pokazuje na segment steka gde se chuva stek programa koji se trenutno
    izvrshava. Za razliku od registra za segment koda, {\eng ss} registar
    se mozhe eksplicitno postaviti shto dozvoljava aplikacijama da postave
    vishe stekova i da alterniraju izmedju njih.
\end{itemize}


\section{{\eng Real mode}}
\medskip

Realni mod je stanje procesora u kojem nam je dozvoljeno adresiranje samo prvih
20 megabajta memorije. Prelazak iz realnog u zastic1eni mod postizhe se dalekim skokom
{\eng "far jump"}.

%{\eng\url{https://wiki.osdev.org/Real_Mode}}

\section{Segmentacija}
\medskip

Segmentacija je reshenje kojim se omoguc1ava adresiranje vishe memorije nego
shto je to hardverski predvidjeno.

%{\eng\url{https://wiki.osdev.org/Segmentation}}

\section{{\eng Protected mode}}
\medskip

Zashtic1en mod je stanje procesora u kojem procesor ima pun pristup celom opsegu
memorije za razliku od realnog moda.

%{\eng\url{https://wiki.osdev.org/Protected_Mode}}

%%%%%%%%%%%%%%%%%%%%%%%%%%%%%%%%%%%%%%%%%%%%%%%%%%%%%%%%%%%%%%%%%%%%%%%%%%%%%%%%
%%%%%%%%%%%%%%%%%%%%%%%%%%%%%%%%%%%%%%%%%%%%%%%%%%%%%%%%%%%%%%%%%%%%%%%%%%%%%%%%
%%%%%%%%%%%%%%%%%%%%%%%%%%%%%%%%%%%%%%%%%%%%%%%%%%%%%%%%%%%%%%%%%%%%%%%%%%%%%%%%
\chapter{{\eng Boot}}

\section{Redosled pokretanja}
\medskip

Od pritiska dugmeta za paljenje rachunara, pa do uchitavanja operativnog sitema
postoji ceo jedan proces. Nakon pritiska dugmeta rachunar prvo izvrshava {\eng
POST (Power On Self Test)} koji je jedna od pochetnih faza {\eng BIOS}-a {\eng
(Basic Input Output System)}. U {\eng POST}-u rachunar pokushava da
incijalizuje komponente rachunarskog sistema i proverava da li one ispunjavaju
sve uslove za startovanje rachunara. Ukoliko je ceo proces proshao bez greshaka
nastavlja se dalje izvrshavanje {\eng BIOS}-a. {\eng BIOS} sada ima ulogu da
pronadje medijum koji sadrzhi program koji c1e uchitati jezgro operativnog
sistema u ram memoriju rachunara. Taj program se naziva {\eng Bootloader}.

%{\eng\url{https://wiki.osdev.org/Boot_Sequence}}

\section{{\eng Bootloader}}
\medskip

{\eng Bootloader} je program koji se nalazi u prvih 512bitova medijuma u {\eng
MBR} odeljku, i njegov zadatak je da uchita jezgro operativnog sistema u ram
memoriju i preda mu dalje upravljanje.

%{\eng\url{https://wiki.osdev.org/Bootloader}}

\section{{\eng Multiboot2}}
\medskip

\section{{\eng ELF}}
\medskip

{\eng ELF} je format binarni fajl koji se sastoji od tachno odredjenih sekcija
i koji mozhe da se pokrene.

%{\eng\url{https://wiki.osdev.org/ELF}}

%%%%%%%%%%%%%%%%%%%%%%%%%%%%%%%%%%%%%%%%%%%%%%%%%%%%%%%%%%%%%%%%%%%%%%%%%%%%%%%%
%%%%%%%%%%%%%%%%%%%%%%%%%%%%%%%%%%%%%%%%%%%%%%%%%%%%%%%%%%%%%%%%%%%%%%%%%%%%%%%%
%%%%%%%%%%%%%%%%%%%%%%%%%%%%%%%%%%%%%%%%%%%%%%%%%%%%%%%%%%%%%%%%%%%%%%%%%%%%%%%%
\chapter{Korish\/c1eni alati}
\bigskip

U daljem tekstu se mogu videti neki od alata korish\/c1enih u kreiranju ovog
rada. Vec1ina korish\/c1enih alata poseduje {\eng GPLv3} licencu.
{\eng GNU Public Licence} je licenca otvorenog koda koja dozvoljava
modifikovanje i distribuiranje koda sve dok taj je taj kod javno dostupan.
Jedini programi sa ove liste koji nije kreirao {\eng GNU} su {\eng QEMU}
virtualna mashina, {\eng git} i {\eng NeoVim}.\\

Operativni sistem korish\/c1en u izradi ovog projekta je {\eng Artix Linux}.
{\eng Artix Linux} je {\eng GNU/Linux} distribucija bazirana na {\eng Arch Linux}-u.
Vec1ina korish\/c1enih programa je vec1 kompajlovana i spremna za upotrebu i
nalazi se u oficijalnim repozitorijima.\\

Za programe koji su morali biti manuelno kompajlovani date su instrukcije u
njihovoj podsekciji. Programi koji su morali biti kompajlovani su {\eng
binutils} i {\eng gcc} i to da ne bi koristili standardnu biblioteku koju nam
je obezbedio operativni sistem domac1in (onaj na kome se kompajluje ovaj
projekat). Za ostale programe koji su korish\/c1eni preporuka je koristiti one
koji su dostupni kao spremni paketi u izvorima odabrane distrubucije {\eng
GNU/Linux}-a.

\section{{\eng Binutils}}
\medskip

Izvorni kod softvera se mozhe nac1i na stranici
{\eng\url{https://www.gnu.org/software/binutils/}},
zajedno sa uput\/stvom za kompajlovanje i korish\/c1enje.

Ovaj softverski paket sadrzhi programe neophodne za izradu programa od kojih su
najkorisheniji asembler ({\eng as}), linker ({\eng ld}), kao i program za
kreiranje biblioteka ({\eng ar}).

\subsection{Pre dodavanja {\eng C} biblioteke}
\smallskip

Iz razloga shto se ne koristi standardna biblioteka vec1 samostalno napisana
specificno za ovaj projekat, potrebno je manuelno kompajlovati {\eng GNU Binutils}.
Medjutim, postoji moguc1nost korish\/c1enja vec1 spremnog paketa koji se za
distribucije bazirane na {\eng Arch Linux}-u mozhe nac1i na stanici
{\eng\url{https://aur.archlinux.org/packages/i686-elf-binutils/}}. Pojedine distribucije
vec1 imaju ovaj paket kompajlovan, ali je preporuka manuelno kompajlovati da bi
se izbegla nekompatibilnost, a i prosto iz razloga shto c1e nakon formiranja
nashe {\eng C} biblioteke biti neophodno kompajlovati ovaj program za svaki
sistem posebno.

Za one koje zhele sami da kompajluju dat je deo instrukcija koji se razlikuje
od uput\/stva datog na zvanichnom sajtu a tiche se konfigurisanja pre kompilacije.\\

\begin{minipage}{\textwidth}\eng\lstinputlisting[language=make]{include/00.alati/binutils/binutils1}\srb\end{minipage}

\subsection{Nakon dodavanja  {\eng C} biblioteke}
\smallskip

Nakon dodavanja nashe {\eng C} biblioteke potrebno je kompajlovati {\eng GNU
Binutils} tako da tu biblioteku i koristi prilikom kompajlovanja nasheg
operativnog sistema.

\textbf{Napomena:} Potrebno je postaviti {\eng \$SYSROOT} na lokaciju gde se biblioteka nalazi.
To je moguc1e uraditi na sledec1i nachin:\\

\begin{minipage}{\textwidth}\eng\lstinputlisting[language=make]{include/00.alati/binutils/exportsysroot}\srb\end{minipage}

Instukcije za kompajlovanje date su u nastavku:

\begin{minipage}{\textwidth}\eng\lstinputlisting[language=make]{include/00.alati/binutils/binutils2}\srb\end{minipage}

\subsection{{\eng GNU Asembler}}
\smallskip
Iako trenutno postoje mnogo popularnije alternative poput {\eng NASM (Netwide
Assembler)} i {\eng MASM (Microsoft Assembler)} koji koriste noviju Intelovu
sintaksu, autor se ipak odluchio za {\eng GASM} zbog kompatibilnosti sa {\eng
GCC} kompajlerom. {\eng GASM} kosristi stariju {\eng AT\&T} sintaksu koju
karakterishe: obrnut poredak parametara, prefiks pre imena registara i
vrednosti konstanti, a i velichina parametara mora biti eksplicitno definisana.
Zbog toga c1e mozhda nekim chitaocima biti koristan program {\eng "intel2gas"}
koji se za {\eng Arch Linux} mozhe nac1i na stanici
{\eng\url{https://aur.archlinux.org/packages/intel2gas/}}.\\

Ovaj program je korish\/c1en za kompajlovanje dela koda napisanog u asembleru.

\subsection{{\eng GNU Linker}}
\smallskip

Ovaj program je korish\/c1en za linkovanje, tj. "spajanje" svog komapjlovanog koda
u jednu binarnu datoteku tipa {\eng ELF} koja predstavlja kernel.

\section{{\eng GCC}}
\medskip

Izvorni kod softvera se mozhe nac1i na stranici
{\eng\url{https://gcc.gnu.org/}},
zajedno sa uput\/stvom za kompajlovanje i korish\/c1enje.


{\eng\url{https://aur.archlinux.org/packages/i686-elf-gcc/}}

{\eng GCC} je {\eng GNU}-ov set kompajlera.

\subsection{Pre dodavanja {\eng LIBC}}
\smallskip

\begin{minipage}{\textwidth}\eng\lstinputlisting[language=make]{include/00.alati/gcc/gcc1}\srb\end{minipage}

\subsection{Posle dodavanja {\eng LIBC}}

\smallskip

\begin{minipage}{\textwidth}\eng\lstinputlisting[language=make]{include/00.alati/gcc/gcc2}\srb\end{minipage}

\section{{\eng GRUB}}
\medskip

Izvorni kod softvera se mozhe nac1i na stranici

{\eng\url{https://www.gnu.org/software/grub/}},

zajedno sa uput\/stvom za kompajlovanje i korish\/c1enje.

{\eng GRUB} je {\eng bootloader} koji je korish\/c1en na ovom projektu. Plan je
da u buduc1nosti {\eng GRUB} bude zamenjen sa {\eng bootloader}-om izradjenim
specificno za ovaj operativni sistem i da sve komponente ovog operativnog
sistema na taj nachin budu delo jednog autora.

\section{{\eng QEMU}}
\medskip

Izvorni kod softvera se mozhe nac1i na stranici
{\eng\url{https://www.qemu.org/}},
zajedno sa uput\/stvom za kompajlovanje i korish\/c1enje.

{\eng QEMU} je virtualna mashina u kojoj c1e jezgro biti testirano i
prikazano zarad praktichnih razloga. {\eng QEMU} je odabran za ovaj projekat
jer za razliku od drugih virutalnih mashina poseduje {\eng cli (command line
interface)} iz koga se lako mozhe pozivati iz skripti kao shto su {\eng Makefile}-ovi.

\section{{\eng Make}}
\medskip

Izvorni kod softvera se mozhe nac1i na stranici

{\eng\url{https://www.gnu.org/software/make/}}

zajedno sa uput\/stvom za kompajlovanje i korish\/c1enje.
\cite{book:78575}.

{\eng Make} nam omoguc1ava da sa lakoc1om odrzhavamo i manipulishemo izvornim
fajlovima. Moguc1e je sve kompajlovati, obrisati, kreirati {\eng iso} fajl kao
i pokrenuti {\eng QEMU} virtuelnu mashinu sa samo jednom kljchnom rechi u
terminalu. Kreirani {\eng Makefile} za potrebe ovog projekta bic1e detaljno
objashnjen u daljem tekstu.

\section{Manje bitni alati}
\medskip

\subsection{{\eng NeoVim}}
{\eng NeoVim} je uredjivach teksta nastao od {\eng Vim}-a ({\eng Vi improved}).
\cite{book:78583}. Konfiguracijski fajlovi autora, mogu se nac1i na
{\eng\url{https://github.com/aleksav013/nvim}}, i imaju za cilj da stvore
okruzhenje pogodno za rad na ovom projektu.

\subsection{{\eng git}}
\smallskip

Kreator ovog programa je {\eng Linus Torvalds}, chovek koji je kreirao {\eng
Linux kernel}.
Izvorni kod softvera se mozhe nac1i na stranici
{\eng\url{https://git.kernel.org/pub/scm/git/git.git}}.

{\eng Git} je program koji nam pomazhe da odrzhavamo izvodne fajlove
sinhronizovanim sa repozitorijimom. Osim toga znachajan je i njegov sistem
"kontrole" verzija - moguc1nost da se za svaki {\eng commit}(promenu) vidi
tachno koji su se fajlovi izmenili i koja je razlika izmedju neke dve verzije
projekta.

\subsection{{\eng xorriso(libisoburn)}}
\smallskip
{\eng\url{https://www.gnu.org/software/xorriso/}}

Sluzhi za kreiranje {\eng ISO} fajlova koji se mogu "narezati" na  {\eng CD}
ili {\eng USB} flesh sa kojih se kasnije dizhe sistem.

\subsection{{\eng GDB}}
\smallskip
{\eng\url{https://www.sourceware.org/gdb/}}

{\eng GNU}-ov {\eng debugger} koji sluzhi uglavnom za pronalazhenje greshaka u
kodu.

%%%%%%%%%%%%%%%%%%%%%%%%%%%%%%%%%%%%%%%%%%%%%%%%%%%%%%%%%%%%%%%%%%%%%%%%%%%%%%%%
%%%%%%%%%%%%%%%%%%%%%%%%%%%%%%%%%%%%%%%%%%%%%%%%%%%%%%%%%%%%%%%%%%%%%%%%%%%%%%%%
%%%%%%%%%%%%%%%%%%%%%%%%%%%%%%%%%%%%%%%%%%%%%%%%%%%%%%%%%%%%%%%%%%%%%%%%%%%%%%%%
\chapter{Inspiracija}
\bigskip

\section{{\eng Minix}}
\medskip
\cite{book:821745}\\
\cite{book:1400099}
Takodje je napisao i \cite{book:915673}.

\section{{\eng Linux}}
\medskip

\section{{\eng BSD}}
\medskip
\cite{book:1310096}

\section{{\eng Mmurtl}}
\medskip

\cite{book:658757}
%%%%%%%%%%%%%%%%%%%%%%%%%%%%%%%%%%%%%%%%%%%%%%%%%%%%%%%%%%%%%%%%%%%%%%%%%%%%%%%%
%%%%%%%%%%%%%%%%%%%%%%%%%%%%%%%%%%%%%%%%%%%%%%%%%%%%%%%%%%%%%%%%%%%%%%%%%%%%%%%%
%%%%%%%%%%%%%%%%%%%%%%%%%%%%%%%%%%%%%%%%%%%%%%%%%%%%%%%%%%%%%%%%%%%%%%%%%%%%%%%%
\chapter{{\eng Build system}}
\bigskip

\section{{\eng aleksa-toolchain}}
\medskip

\section{{\eng Makefile}}
\medskip

%%%%%%%%%%%%%%%%%%%%%%%%%%%%%%%%%%%%%%%%%%%%%%%%%%%%%%%%%%%%%%%%%%%%%%%%%%%%%%%%
%%%%%%%%%%%%%%%%%%%%%%%%%%%%%%%%%%%%%%%%%%%%%%%%%%%%%%%%%%%%%%%%%%%%%%%%%%%%%%%%
%%%%%%%%%%%%%%%%%%%%%%%%%%%%%%%%%%%%%%%%%%%%%%%%%%%%%%%%%%%%%%%%%%%%%%%%%%%%%%%%
\chapter{Jezgro operativnog sistema}
\bigskip

Rad je prvobitno bio zamisljen kao postupno izlaganje nastajanja ovog
operativnog sistema, ali se kasnije autor odluchio da ipak izlozhi samo
trenutnu verziju rada, s obrzirom na to da bi rad bio nepotrebno duzhi.

\section{Pochetak}
\medskip
%{\eng\url{https://wiki.osdev.org/Bare_Bones}}

{\eng as/boot.s}:

U prvom delu postavljamo promenljive na vrednosti koje su odredjene {\eng
Multiboot2} standardom da bi {\eng bootloader} prepoznao nashe jezgro.

\begin{minipage}{\textwidth}\eng\lstinputlisting[language=Assembler]{include/01.pocetak/deo1}\srb\end{minipage}

Nakon toga postavljamo prvih 512 bitova na prethodno pomenute vrednosti ali
tako da za svaku promenljivu ostavljamo 32 bita prostora.

\begin{minipage}{\textwidth}\eng\lstinputlisting[language=Assembler]{include/01.pocetak/deo2}\srb\end{minipage}

Postavljamo funkcije koje cemo definisati u ovom fajlu za globalne da bi smo
kasnije mogli da ih pozivamo iz {\eng C}-a.

\begin{minipage}{\textwidth}\eng\lstinputlisting[language=Assembler]{include/01.pocetak/deo3}\srb\end{minipage}

Funkcija za uchitavanje {\eng gdt} tabele.

\begin{minipage}{\textwidth}\eng\lstinputlisting[language=Assembler]{include/01.pocetak/deo4}\srb\end{minipage}

Funkcija za uchitavanje {\eng idt} tabele.

\begin{minipage}{\textwidth}\eng\lstinputlisting[language=Assembler]{include/01.pocetak/deo5}\srb\end{minipage}

Funkcije koje su zaduzene za razmenu informacija preko magistrale za
ulaz/izlaz. Koristi se pri inicijalizaciji {\eng IRQ}-a i korish\/c1enju
tastature.

\begin{minipage}{\textwidth}\eng\lstinputlisting[language=Assembler]{include/01.pocetak/deo6}\srb\end{minipage}
\begin{minipage}{\textwidth}\eng\lstinputlisting[language=Assembler]{include/01.pocetak/deo7}\srb\end{minipage}

Segmenti za kod i podatke koji su postavljeni u {\eng gdt} tabeli.

\begin{minipage}{\textwidth}\eng\lstinputlisting[language=Assembler]{include/01.pocetak/deo8}\srb\end{minipage}

Definishemo sekciju {\eng bss} u kojoj kreiramo stek i dodeljujemo mu 16
kilobajta.

\begin{minipage}{\textwidth}\eng\lstinputlisting[language=Assembler]{include/01.pocetak/deo9}\srb\end{minipage}

Definishemo pochetnu funkciju {\eng \_start} pozivajuc1i funkciju za
inicijalizaciju {\eng gdt} tabele i "skachemo" na segment koda. Ovaj postupak
ima naziv {\eng "far jump"} jer skachemo van tekuc1eg segmenta.

\begin{minipage}{\textwidth}\eng\lstinputlisting[language=Assembler]{include/01.pocetak/deo10}\srb\end{minipage}

U segmentu koda postavljamo segmentne registre na adresu segmenta podataka.
Zatim postavljamo {\eng esp} registar na pochetak steka koji smo inicijalizovali
u {\eng bss} sekciji i predajemo upravljanje {\eng kernel\_main} funkciji.

\begin{minipage}{\textwidth}\eng\lstinputlisting[language=Assembler]{include/01.pocetak/deo11}\srb\end{minipage}

Postavljamo velichinu funkcije {\eng \_start} shto nam kasnije mozhe biti
korisno pri {\eng debug}-ovanju.

\begin{minipage}{\textwidth}\eng\lstinputlisting[language=Assembler]{include/01.pocetak/deo12}\srb\end{minipage}



\section{Ispis na ekran - {\eng VGA}}
\medskip

{\eng c/vga.c}:

\begin{minipage}{\textwidth}\eng\lstinputlisting[language=C]{include/02.vga/deo1}\srb\end{minipage}

Primetimo da u {\eng C}-u koristimo {\eng uintX\_t} promenljive. To je zbog toga
shto nam je u ovakvom okruzhenju vrlo bitno da pazimo na velichinu koju
zauzimaju promenljive.

\begin{minipage}{\textwidth}\eng\lstinputlisting[language=C]{include/02.vga/deo2}\srb\end{minipage}

4 znachajnija bita oznachavaju boju pozadine, dok ostala 4 bita oznachavaju boju karaktera.

\begin{minipage}{\textwidth}\eng\lstinputlisting[language=C]{include/02.vga/deo3}\srb\end{minipage}

\begin{minipage}{\textwidth}\eng\lstinputlisting[language=C]{include/02.vga/deo4}\srb\end{minipage}

Na {\eng VGA} izlaz ispisujemo tako shto pochev od adrese {\eng 0xB80000}
pishemo shesnaestobitne vrednosti koje se prevode u karaktere i njihovu boju. 8
znachajnijih bitova odredjuju boju karaktera dok preostalih 8 bitova oznachavaju
karakter.

\begin{minipage}{\textwidth}\eng\lstinputlisting[language=C]{include/02.vga/deo5}\srb\end{minipage}

Funkcija koja ispisuje karakter na monitoru.

\begin{minipage}{\textwidth}\eng\lstinputlisting[language=C]{include/02.vga/deo6}\srb\end{minipage}

Fukcija koja pomera sve do sada ispisano za jedan red na dole i oslobadja novi
red kada ponestane mesta na ekranu.

\begin{minipage}{\textwidth}\eng\lstinputlisting[language=C]{include/02.vga/deo7}\srb\end{minipage}

Funkfija koja postavlja brojache kolone i reda na sledec1e, uglavnom prazno, polje na ekranu.

\begin{minipage}{\textwidth}\eng\lstinputlisting[language=C]{include/02.vga/deo8}\srb\end{minipage}

Funkfija koja postavlja brojache kolone i reda na proshlo polje na ekranu.

\begin{minipage}{\textwidth}\eng\lstinputlisting[language=C]{include/02.vga/deo9}\srb\end{minipage}

Funkcija koja ispisuje jedan karakter na ekran. Proverava da li je potrebno
ispisati novi red umesto karaktera {\eng '$\backslash$n'}, kao i da li je
potrebno osloboditi novi red ukoliko se ekran popunio.

\begin{minipage}{\textwidth}\eng\lstinputlisting[language=C]{include/02.vga/deo10}\srb\end{minipage}

Funkcija koja ispisuje niz karaktera na ekran.

\begin{minipage}{\textwidth}\eng\lstinputlisting[language=C]{include/02.vga/deo11}\srb\end{minipage}

Funcija koja ispisuje celobrojnu vrednost na ekran tako shto je prvo pretvori u niz
karaktera a zatim iskoristi prethodnu funkciju.

\begin{minipage}{\textwidth}\eng\lstinputlisting[language=C]{include/02.vga/deo12}\srb\end{minipage}

Funcija koja ispisuje realnu vrednost na ekran tako shto je prvo pretvori u niz
karaktera a zatim iskoristi funkciju za ispis niza karaktera.

\begin{minipage}{\textwidth}\eng\lstinputlisting[language=C]{include/02.vga/deo13}\srb\end{minipage}

Funkcija koja brishe sve sa ekrana i postavlja brojache kolone i reda na pochetnu poziciju.

\begin{minipage}{\textwidth}\eng\lstinputlisting[language=C]{include/02.vga/deo14}\srb\end{minipage}


\section{{\eng Global Desctiptor Table}}
\medskip

{\eng c/gdt.c}:

\begin{minipage}{\textwidth}\eng\lstinputlisting[language=C]{include/03.gdt/deo1}\srb\end{minipage}

Format u kom rachunar prihvata unos pojedinachnih definicija segmenata.
Primetimo {\eng \_\_attribute\_\_((packed))}, na kraju definicije strukture. To
nam oznachava da se nece ostavljati mesta u memoriji izmedju promenljivih
unutar strukture, vec1 c1e se "pakovati" jedna do druge u memoriji.

\begin{minipage}{\textwidth}\eng\lstinputlisting[language=C]{include/03.gdt/deo2}\srb\end{minipage}

Format koji rachunar prihvata za tabelu svih definicija segmenata.

\begin{minipage}{\textwidth}\eng\lstinputlisting[language=C]{include/03.gdt/deo3}\srb\end{minipage}

Funcija iz asemblera koja uchitava tabelu segmenata, kreiranu u sledec1ih
nekoliko funcija, u odgovarajuc1i registar. Ovu funciju smo imali priliku
videti u pochetnom fajlu.

\begin{minipage}{\textwidth}\eng\lstinputlisting[language=C]{include/03.gdt/deo4}\srb\end{minipage}


\begin{minipage}{\textwidth}\eng\lstinputlisting[language=C]{include/03.gdt/deo5}\srb\end{minipage}


\begin{minipage}{\textwidth}\eng\lstinputlisting[language=C]{include/03.gdt/deo6}\srb\end{minipage}


\begin{minipage}{\textwidth}\eng\lstinputlisting[language=C]{include/03.gdt/deo7}\srb\end{minipage}


\begin{minipage}{\textwidth}\eng\lstinputlisting[language=C]{include/03.gdt/deo8}\srb\end{minipage}


\begin{minipage}{\textwidth}\eng\lstinputlisting[language=C]{include/03.gdt/deo9}\srb\end{minipage}

%{\eng\url{https://wiki.osdev.org/GDT}}

\section{{\eng Interrupt Desctiptor Table}}
\medskip

{\eng c/idt.c}:

\begin{minipage}{\textwidth}\eng\lstinputlisting[language=C]{include/04.idt/deo1}\srb\end{minipage}
\begin{minipage}{\textwidth}\eng\lstinputlisting[language=C]{include/04.idt/deo2}\srb\end{minipage}
\begin{minipage}{\textwidth}\eng\lstinputlisting[language=C]{include/04.idt/deo3}\srb\end{minipage}
\begin{minipage}{\textwidth}\eng\lstinputlisting[language=C]{include/04.idt/deo4}\srb\end{minipage}
\begin{minipage}{\textwidth}\eng\lstinputlisting[language=C]{include/04.idt/deo5}\srb\end{minipage}
\begin{minipage}{\textwidth}\eng\lstinputlisting[language=C]{include/04.idt/deo6}\srb\end{minipage}
\begin{minipage}{\textwidth}\eng\lstinputlisting[language=C]{include/04.idt/deo7}\srb\end{minipage}
\begin{minipage}{\textwidth}\eng\lstinputlisting[language=C]{include/04.idt/deo8}\srb\end{minipage}
\begin{minipage}{\textwidth}\eng\lstinputlisting[language=C]{include/04.idt/deo9}\srb\end{minipage}
\begin{minipage}{\textwidth}\eng\lstinputlisting[language=C]{include/04.idt/deo10}\srb\end{minipage}
\begin{minipage}{\textwidth}\eng\lstinputlisting[language=C]{include/04.idt/deo11}\srb\end{minipage}

%{\eng\url{https://wiki.osdev.org/IDT}}

\section{{\eng IRQ} i {\eng PIC}}
\medskip
{\eng c/idt.c}:

\begin{minipage}{\textwidth}\eng\lstinputlisting[language=C]{include/04.idt/deo12}\srb\end{minipage}
\begin{minipage}{\textwidth}\eng\lstinputlisting[language=C]{include/04.idt/deo13}\srb\end{minipage}
\begin{minipage}{\textwidth}\eng\lstinputlisting[language=C]{include/04.idt/deo14}\srb\end{minipage}


%{\eng\url{https://wiki.osdev.org/IRQ}}
%{\eng\url{https://wiki.osdev.org/PIC}}

\section{Tastatura}
\medskip

{\eng c/keyboard.c}:

\begin{minipage}{\textwidth}\eng\lstinputlisting[language=C]{include/06.keyboard/deo1}\srb\end{minipage}
\begin{minipage}{\textwidth}\eng\lstinputlisting[language=C]{include/06.keyboard/deo2}\srb\end{minipage}
\begin{minipage}{\textwidth}\eng\lstinputlisting[language=C]{include/06.keyboard/deo3}\srb\end{minipage}
\begin{minipage}{\textwidth}\eng\lstinputlisting[language=C]{include/06.keyboard/deo4}\srb\end{minipage}
\begin{minipage}{\textwidth}\eng\lstinputlisting[language=C]{include/06.keyboard/deo5}\srb\end{minipage}
\begin{minipage}{\textwidth}\eng\lstinputlisting[language=C]{include/06.keyboard/deo6}\srb\end{minipage}
\begin{minipage}{\textwidth}\eng\lstinputlisting[language=C]{include/06.keyboard/deo7}\srb\end{minipage}
\begin{minipage}{\textwidth}\eng\lstinputlisting[language=C]{include/06.keyboard/deo8}\srb\end{minipage}
\begin{minipage}{\textwidth}\eng\lstinputlisting[language=C]{include/06.keyboard/deo9}\srb\end{minipage}
\begin{minipage}{\textwidth}\eng\lstinputlisting[language=C]{include/06.keyboard/deo10}\srb\end{minipage}
\begin{minipage}{\textwidth}\eng\lstinputlisting[language=C]{include/06.keyboard/deo11}\srb\end{minipage}
\begin{minipage}{\textwidth}\eng\lstinputlisting[language=C]{include/06.keyboard/deo12}\srb\end{minipage}
\begin{minipage}{\textwidth}\eng\lstinputlisting[language=C]{include/06.keyboard/deo13}\srb\end{minipage}
\begin{minipage}{\textwidth}\eng\lstinputlisting[language=C]{include/06.keyboard/deo14}\srb\end{minipage}
\begin{minipage}{\textwidth}\eng\lstinputlisting[language=C]{include/06.keyboard/deo15}\srb\end{minipage}
\begin{minipage}{\textwidth}\eng\lstinputlisting[language=C]{include/06.keyboard/deo16}\srb\end{minipage}
\begin{minipage}{\textwidth}\eng\lstinputlisting[language=C]{include/06.keyboard/deo17}\srb\end{minipage}
\begin{minipage}{\textwidth}\eng\lstinputlisting[language=C]{include/06.keyboard/deo18}\srb\end{minipage}
\begin{minipage}{\textwidth}\eng\lstinputlisting[language=C]{include/06.keyboard/deo19}\srb\end{minipage}
\begin{minipage}{\textwidth}\eng\lstinputlisting[language=C]{include/06.keyboard/deo20}\srb\end{minipage}

\section{{\eng PIT - Programmable Interval Timer}}
\medskip

{\eng c/timer.c}:

\begin{minipage}{\textwidth}\eng\lstinputlisting[language=C]{include/07.pit/deo1}\srb\end{minipage}
\begin{minipage}{\textwidth}\eng\lstinputlisting[language=C]{include/07.pit/deo2}\srb\end{minipage}
\begin{minipage}{\textwidth}\eng\lstinputlisting[language=C]{include/07.pit/deo3}\srb\end{minipage}
\begin{minipage}{\textwidth}\eng\lstinputlisting[language=C]{include/07.pit/deo4}\srb\end{minipage}
\begin{minipage}{\textwidth}\eng\lstinputlisting[language=C]{include/07.pit/deo5}\srb\end{minipage}
\begin{minipage}{\textwidth}\eng\lstinputlisting[language=C]{include/07.pit/deo6}\srb\end{minipage}
\begin{minipage}{\textwidth}\eng\lstinputlisting[language=C]{include/07.pit/deo7}\srb\end{minipage}
\begin{minipage}{\textwidth}\eng\lstinputlisting[language=C]{include/07.pit/deo8}\srb\end{minipage}
\begin{minipage}{\textwidth}\eng\lstinputlisting[language=C]{include/07.pit/deo9}\srb\end{minipage}
\begin{minipage}{\textwidth}\eng\lstinputlisting[language=C]{include/07.pit/deo10}\srb\end{minipage}
%{\eng\url{https://wiki.osdev.org/PIT}}

\section{{\eng Heap}}
\medskip

{\eng c/heap.c}:

\cite{book:1412}\\

\begin{minipage}{\textwidth}\eng\lstinputlisting[language=C]{include/08.heap/deo1}\srb\end{minipage}
\begin{minipage}{\textwidth}\eng\lstinputlisting[language=C]{include/08.heap/deo2}\srb\end{minipage}
\begin{minipage}{\textwidth}\eng\lstinputlisting[language=C]{include/08.heap/deo3}\srb\end{minipage}
\begin{minipage}{\textwidth}\eng\lstinputlisting[language=C]{include/08.heap/deo4}\srb\end{minipage}
\begin{minipage}{\textwidth}\eng\lstinputlisting[language=C]{include/08.heap/deo5}\srb\end{minipage}
\begin{minipage}{\textwidth}\eng\lstinputlisting[language=C]{include/08.heap/deo6}\srb\end{minipage}
\begin{minipage}{\textwidth}\eng\lstinputlisting[language=C]{include/08.heap/deo7}\srb\end{minipage}
\begin{minipage}{\textwidth}\eng\lstinputlisting[language=C]{include/08.heap/deo8}\srb\end{minipage}
\begin{minipage}{\textwidth}\eng\lstinputlisting[language=C]{include/08.heap/deo9}\srb\end{minipage}
\begin{minipage}{\textwidth}\eng\lstinputlisting[language=C]{include/08.heap/deo10}\srb\end{minipage}
\begin{minipage}{\textwidth}\eng\lstinputlisting[language=C]{include/08.heap/deo11}\srb\end{minipage}
\begin{minipage}{\textwidth}\eng\lstinputlisting[language=C]{include/08.heap/deo12}\srb\end{minipage}
\begin{minipage}{\textwidth}\eng\lstinputlisting[language=C]{include/08.heap/deo13}\srb\end{minipage}
\begin{minipage}{\textwidth}\eng\lstinputlisting[language=C]{include/08.heap/deo14}\srb\end{minipage}
\begin{minipage}{\textwidth}\eng\lstinputlisting[language=C]{include/08.heap/deo15}\srb\end{minipage}
\begin{minipage}{\textwidth}\eng\lstinputlisting[language=C]{include/08.heap/deo16}\srb\end{minipage}
\begin{minipage}{\textwidth}\eng\lstinputlisting[language=C]{include/08.heap/deo17}\srb\end{minipage}
\begin{minipage}{\textwidth}\eng\lstinputlisting[language=C]{include/08.heap/deo18}\srb\end{minipage}
\begin{minipage}{\textwidth}\eng\lstinputlisting[language=C]{include/08.heap/deo19}\srb\end{minipage}
\begin{minipage}{\textwidth}\eng\lstinputlisting[language=C]{include/08.heap/deo20}\srb\end{minipage}
\begin{minipage}{\textwidth}\eng\lstinputlisting[language=C]{include/08.heap/deo21}\srb\end{minipage}
\begin{minipage}{\textwidth}\eng\lstinputlisting[language=C]{include/08.heap/deo22}\srb\end{minipage}
\begin{minipage}{\textwidth}\eng\lstinputlisting[language=C]{include/08.heap/deo23}\srb\end{minipage}
\begin{minipage}{\textwidth}\eng\lstinputlisting[language=C]{include/08.heap/deo24}\srb\end{minipage}
\begin{minipage}{\textwidth}\eng\lstinputlisting[language=C]{include/08.heap/deo25}\srb\end{minipage}
\begin{minipage}{\textwidth}\eng\lstinputlisting[language=C]{include/08.heap/deo26}\srb\end{minipage}
\begin{minipage}{\textwidth}\eng\lstinputlisting[language=C]{include/08.heap/deo27}\srb\end{minipage}
\begin{minipage}{\textwidth}\eng\lstinputlisting[language=C]{include/08.heap/deo28}\srb\end{minipage}
\begin{minipage}{\textwidth}\eng\lstinputlisting[language=C]{include/08.heap/deo29}\srb\end{minipage}
\begin{minipage}{\textwidth}\eng\lstinputlisting[language=C]{include/08.heap/deo30}\srb\end{minipage}
\begin{minipage}{\textwidth}\eng\lstinputlisting[language=C]{include/08.heap/deo31}\srb\end{minipage}
%{\eng\url{https://wiki.osdev.org/Heap}}

\section{{\eng Paging}}
\medskip

{\eng c/paging.c}:

\begin{minipage}{\textwidth}\eng\lstinputlisting[language=C]{include/09.paging/deo1}\srb\end{minipage}
\begin{minipage}{\textwidth}\eng\lstinputlisting[language=C]{include/09.paging/deo2}\srb\end{minipage}
\begin{minipage}{\textwidth}\eng\lstinputlisting[language=C]{include/09.paging/deo3}\srb\end{minipage}
\begin{minipage}{\textwidth}\eng\lstinputlisting[language=C]{include/09.paging/deo4}\srb\end{minipage}
\begin{minipage}{\textwidth}\eng\lstinputlisting[language=C]{include/09.paging/deo5}\srb\end{minipage}
\begin{minipage}{\textwidth}\eng\lstinputlisting[language=C]{include/09.paging/deo6}\srb\end{minipage}
\begin{minipage}{\textwidth}\eng\lstinputlisting[language=C]{include/09.paging/deo7}\srb\end{minipage}
\begin{minipage}{\textwidth}\eng\lstinputlisting[language=C]{include/09.paging/deo8}\srb\end{minipage}
\begin{minipage}{\textwidth}\eng\lstinputlisting[language=C]{include/09.paging/deo9}\srb\end{minipage}

%{\eng\url{https://wiki.osdev.org/Paging}}

\section{Moj {\eng LIBC}}
\medskip
%{\eng\url{https://wiki.osdev.org/Creating_a_C_Library}}

{\eng include/asm.h}:

\begin{minipage}{\textwidth}\eng\lstinputlisting[language=C]{include/10.libc/asm.h}\srb\end{minipage}

{\eng include/errno.h}:

\begin{minipage}{\textwidth}\eng\lstinputlisting[language=C]{include/10.libc/errno.h}\srb\end{minipage}

{\eng include/heap.h}:

\begin{minipage}{\textwidth}\eng\lstinputlisting[language=C]{include/10.libc/heap.h}\srb\end{minipage}

{\eng include/irq.h}:

\begin{minipage}{\textwidth}\eng\lstinputlisting[language=C,linerange={1-5,36-40}]{include/10.libc/irq.h}\srb\end{minipage}

{\eng include/stdio.h}:

\begin{minipage}{\textwidth}\eng\lstinputlisting[language=C]{include/10.libc/stdio.h}\srb\end{minipage}

{\eng include/stdlib.h}:

\begin{minipage}{\textwidth}\eng\lstinputlisting[language=C]{include/10.libc/stdlib.h}\srb\end{minipage}

{\eng include/string.h}:

\begin{minipage}{\textwidth}\eng\lstinputlisting[language=C]{include/10.libc/string.h}\srb\end{minipage}

{\eng include/time.h}:

\begin{minipage}{\textwidth}\eng\lstinputlisting[language=C]{include/10.libc/time.h}\srb\end{minipage}

{\eng include/types.h}:

\begin{minipage}{\textwidth}\eng\lstinputlisting[language=C]{include/10.libc/types.h}\srb\end{minipage}

{\eng include/unistd.h}:

\begin{minipage}{\textwidth}\eng\lstinputlisting[language=C]{include/10.libc/unistd.h}\srb\end{minipage}

{\eng include/vga.h}:

\begin{minipage}{\textwidth}\eng\lstinputlisting[language=C]{include/10.libc/vga.h}\srb\end{minipage}

{\eng include/sys/types.h}:

\begin{minipage}{\textwidth}\eng\lstinputlisting[language=C]{include/10.libc/sys/types.h}\srb\end{minipage}



%%%%%%%%%%%%%%%%%%%%%%%%%%%%%%%%%%%%%%%%%%%%%%%%%%%%%%%%%%%%%%%%%%%%%%%%%%%%%%%%
%%%%%%%%%%%%%%%%%%%%%%%%%%%%%%%%%%%%%%%%%%%%%%%%%%%%%%%%%%%%%%%%%%%%%%%%%%%%%%%%
%%%%%%%%%%%%%%%%%%%%%%%%%%%%%%%%%%%%%%%%%%%%%%%%%%%%%%%%%%%%%%%%%%%%%%%%%%%%%%%%
\chapter{Zakljuchak}
Ovaj projekat je bio sjajan pokazatelj koliko je zapravo kompleksna izrada
jezgra operativnog sistema koji treba da predstavlja most izmedju hardvera i
softvera. Drago mi je shto sam odabrao ovako tezhak projekat za maturski rad iz
razloga shto mi je to pomoglo da probijem barijeru i ulozhim puno truda da bih
zapravo razumeo kako rade operativni sistemi i koliko je sofisticiran njihov
dizajn.

\thispagestyle{empty}
\mbox{}
\clearpage

\nocite{*}
\printbibliography[heading=bibintoc,title={Literatura}]

\end{document}
